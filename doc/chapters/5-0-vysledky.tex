
\chapter{Výsledky a testovanie}

\begin{comment}
    - ukazat historiu trenovania
    - prehladne tabulky vysledkov
    - prehladna tabulka vysledkov pre detekciu uhla s roznymi thresholdami
        * prehladna tabulka pre klasifikaciu zbrani
        * stlpce bude preprocessing a bude hviezdicka alebo pomlcka ak sa to (ne)pouzilo
        * prvy stlpec bude meno klasifikatora popr. nejaka jeho konfiguracia (pri roznych kofiguraciach, urobit rozne tabuky)
        * na koniec stlpec, presnost siete (celkova ale aj pre kazdu kategoriu)
\end{comment}

\begin{comment}
    - Pridat ake optimalizatory a celkovo s akymi argumetmi spustat trenovanie, kolko epoch a pod...
    - Zdovodnovat preco prave taketo nastavenie konvolucnej siete.
    - pocet epoch bude potrebne zistit experimentalnych sposobom, alebo pouzit funkciu na save best of Keras-u
    - pre zabezpecenie spravneho trenovanie boli pocty obrazkov rovnomerne pri trenovani
\end{comment}

\chapter{Úvod}
V~dnešnej dobe sa so strelnými zbraňami stretávame pomerne často, či sú to už rôzne akčné počítačové hry,
    voľno časové aktivity ako airsoft či paintball, ale taktiež ich môžeme denno-denne vidieť na uliciach u~príslušníkov bezpečnostých zložiek štátu.
Z~tohto hľadiska je bezpečnosť veľmi dôležitý prvok nášho života a zbrane k~tomu v~istej prieme prispievajú,
    rovnako ako v~pozívnom ale aj v~negatívnom smere.

Hlavne kvôli negatívnemu vplyvu zbraní sa bežpečnostné zložky snažia znižovať toto riziko a dopad na ľudské zdravie.
Je možné s~určitosťou povedať že technológie, ktoré by dokázali vyhľadávať zbrane u~ľudí na uliciach alebo iných miestach,
    by určite pomohli zvýšiť bezpečnosť danej oblasti alebo krajiny.
V~tomto prípade dokážu počítačové technológie a počítačove videnie pomôcť.

Preto sa v~tejto bakalárskej práci budeme konkrétne venovať určovaniu typu strelnej zbrane v~dvoch kategóriách a to na
    krátke a dlhé zbrane, a určeniu ich náklonu v~obrazovej scéne.

V~prvej kapitole sa zameriame na rozdelenie zbraní podľa rôznych kritérií, následne si uvedieme niekoľko
    významných prístupov k~detekcií a klasifikácií objektu v~obraze, ktoré vrámci spracovania obrazu vytvorili istý prevrat.
A~uvedieme si niekoľko nástrojov ktoré takého spracovanie obrazu uľahčujú.

V~druhej kapitole si následne opíšeme podrobný návrh postupov, ktoré v~tejto práci budú implementované a otestované.
Spôsoby a možnosti získavania dát pre spracovanie obrazu.

Ďalej sa v~tretej kapitole budeme venovať presnému opisu implementácie navrhovaného riešenia.

A~následne sa v~posledných kapitolách pozrieme na výsledky, porovnáme navrhnuté riešenia a zhodnotíme možný budúci postup tejto práce.


\section{Zbrane}

Obvýklá definícia hovorí, že zbraň je nástroj, predmet, či dokonca celé zariadenie,
ktoré je prispôsobené k vyvolaniu zranenia na živý organizmus alebo k ničeniu objektu\cite{book:StrelneZbrane}.
Za prvé zbrane môžeme považovať kopije ktoré používali luďia pri love zvierat asi pred 400,000 rokmi\cite{prop:SpearHistory}.

Vo všeobecnosti môžeme zbrane rozdeliť podľa množstvá krytérií, napr. podľa zdroja energie použitej k vypudeniu projektilu zo zbrane,
podľa konštrukcie a režimu streľby, ďalej z hľadiska postupu pri nabíjaní alebo podľa veku zbrane na nové - slúžiace svojmu účelu a historické - ktoré sú už nespôsobilé k pôvodnemu účelu.
My sa zameriame na 2 základne rozdelenia a to poďla toho ako zbrane pôsobia na živú silu, delíme na\cite{book:StrelneZbrane}:
\begin{enumerate}
	\item[$\bullet$] \textbf{Strelné} - rozrušujú vzdialený cieľ, živý alebo neživý, prodstredníctvom dopadovej energie strely vypudenej zo zbrane,
	\item[$\bullet$] \textbf{Chladné} - účinkujú bodom alebo sekom naostrenej čepele, ktorá je vsadená do rukoväťe alebo je nasadená na tyč či porísko,
    \item[$\bullet$] \textbf{Úderné} - pôsobia na živý objekt tupým úderom svojej časti, ktorá býva spojená s vhodnou rukoväťou,
\end{enumerate}
a podľa ovládateľnosti a možnosti prenášania ich delíme na\cite{book:StrelneZbrane}:
\begin{enumerate}
	\item[$\bullet$] \textbf{Ručné} strelné zbrane môže prenášať a ovládať jediná osoba. Sú ovládané buď jednou rukou - krátke zbrane - alebo oboma rukami - dlhé zbrane,
	\item[$\bullet$] \textbf{Lafetované} zbrane musia byť vzhľadom ku svojej hmotnosti a rozmerom umiestnené na zvláštnom podstavci - \textit{lafete}. Takúto zbraň takisto väčšinou obsluhuje viac ľudí.
\end{enumerate}
V tejto práci sa zameramé na strelné a ručné zbrane s rozdelením na krátke a dlhé.

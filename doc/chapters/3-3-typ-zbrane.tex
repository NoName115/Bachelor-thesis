
\section{Klasifikácia typu zbrane}
- predspracovanie ake chcem pouzit, cely chain vysvetlit
- Pridat ake optimalizatory a celkovo s akymi argumetmi spustat trenovanie, kolko epoch a pod...
- Zdovodnovat preco prave taketo nastavenie konvolucnej siete.

- spomenut ktore triedy zo scikit-learn chcem pouzit pre implementovanie klasifikatorov.
- Budem pouzivam SVM, Kmenas a Multilayer Perceptron
- Opisat postup toho ako to budem robit, chcem vyskusat klasicky pristup HOG + SVM/KMeans
- Nasledne urobit len zakladny preprocessing pre NN a urobit 2 architektury CNN ktore porovnam

- urobit diagram celkoveho postupu ako to bude fungovat (popisat to cele asi radsej ako diagram)
  preprocessing --> Kmeans/SVM --> trenovanie --> vysledok
- napisat ktore trieda implementuje augmentaciu dat, ktore funkcie z akej veci chcem na co pouzit.
  Grayscale, HOG, zo scikit-image. CNN z Keras. Normalizacia NumPy. Augmentacia dat, 

\begin{comment}
  \section{scikit-learn}
    \label{sec:scikitlearn}

    Scikit-learn je softvérová knižica pre strojové učenie pre programovací jazyk Python.
    Obsahuje množstvo algoritmov pre klasifikáciu, regresiu alebo zhlukovanie dát \cite{odkaz:scikitlearn}.
    Pre riešenie tejto práce obsahuje vhodné triedy, ktoré implementujú spomínané postupy zo sekcie \ref{sec:klasifikacia}.
    Príklady tried pre jednotlivé algoritmy:
    \begin{enumerate}
        \item[$\bullet$] \textbf{Nearest Neighbors} - scikit-learn poskytuje 2 rôzne klasifikátory pre algoritmus najbližsieho suseda.
        Trieda \textit{KNeighborsClassifier} klasifikuje na základe $k$ najbližšich susedov, kde $k$ je celé číslo špecifikované užívateľom.
        Druhá trieda \textit{RadiusNeighborsClassifier} implementuje klasifikáciu na základe počtu susedov v rámci pevného polomeru $r$ každého trénovacieho bodu,
        kde $r$ je hodnota s pohyblivou desatinou čiarkou určená užívateľom\footnote{\url{http://scikit-learn.org/stable/modules/neighbors.html\#nearest-neighbors-classification}}.
        \item[$\bullet$] \textbf{Support Vector Machines} - \textit{SVC}, \textit{NuSVC} a \textit{LinearSVC} sú triedy pre viac-triednu klasifikáciu.
        Pre ktoré je možné použit rôzne typy jadier[eng. kernels] \footnote{\url{http://scikit-learn.org/stable/modules/svm.html\#custom-kernels}}.
        \item[$\bullet$] \textbf{Stochastic Gradient Descent} - \textit{SGDClassifier} podporuje viac-triednu klasifikáciu pomocou kombinácie viacerých binárnych klasifikátorov v tzv.“one versus all” (OVA) schéme \footnote{\url{http://scikit-learn.org/stable/modules/sgd.html\#stochastic-gradient-descent}}.
    \end{enumerate}

\end{comment}

\subsection{Trenovanie klasifikatorov}
- pocet epoch bude potrebne zistit experimentalnych sposobom, alebo pouzit funkciu na save best of Keras-u

\subsection{Cross-entropy classification}


\chapter{Záver}

V~práci bolo navrhnutých niekoľko spôsobov pre klasifikovanie typu a určenie náklonu zbrane, od klasických prístupov,
    ktoré využívaju tranformáciu obrazu do šedotónového obrazu a hoghovu transformáciu obrazu pomocou
    klasifikatórov ako K-Nearest-Neighbour a Support Vector Machine.

Až po využitie moderných spôsobov spracovania obrazu, ako sú konvolučné neurónové siete, kde boli navrhnuté
    dve architektúry týchto sietí.

Dosiahnuté výsledky boli porovnané a následne vybrané najlepšie riešenie s~presnosťou klasifikácia zbrane až 83.14\%
    pomocou konvolučnej neurónovej siete.
Taktiež boli dosiahnuté vysoké percentá úspešnosti pre určenie náklonu zbrane použitím konvolučnej neurónovej siete od
    najhoršieho výsledku 54.64\% v~osi yaw až po presnosť 92.34\% v~osi pitch a 95.51\% v~osi roll.

Pri klasifikácii krátkych zbraní ktoré obsahovali tlmič bolo zistené nesprávne určovanie kategórie, avšak po
    konzultáciach s~vedúcim práce sme sa rozhodli tento problém neriešiť a tak tento problém môže byť jeden z bodov
    budúceho vývoja tejto práce.
Taktiež pre budúci vývoj práce by bolo možné rožšíriť veľkosť vstupnej databázy dát a využiť pre klasifikáciu
    už predtrénované hlboké konvolučné neurónové siete.

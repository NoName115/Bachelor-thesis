
\chapter{Záver}

Cieľom práce bolo navrhnúť spôsob pre klasifikáciu typu zbrane do dvoch kategórií na krátke a dlhé, a
    určenie náklonu zbrane v obrazovej scéne.

Vo výsledku bolo navrhnutých niekoľko spôsobov pre riešenie týchto problémov, od klasického prístupu,
    ktorý využíva tranformáciu obrazu do šedotónového obrazu a hoghovu transformáciu obrazu pomocou
    klasifikatórov ako K-Nearest-Neighbour a Support Vector Machine.

Až po využitie moderných spôsobov spracovania obrazu, ako sú konvolučné neurónové siete, kde boli navrhnuté
    dve architektúry týchto sietí.

Tieto navrhované riešenia boli implementované a v závere práce boli podrobne opísané výsledky týchto postupov a
    vybrané najlepšie riešenie.
Následne boli opísané postupy pre možný budúci vývoj tejto práce.

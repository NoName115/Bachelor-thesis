
\chapter{Implementácia a testovanie}

% TODO uvod

\section{Trénovacie dáta}
- loader.py - DataLoader a DataSaver
- ake realne počty dát mam
- urobit tabulky ze ake typy obrazkov a pocty su pre ktoru vec
- ktore triedy implementujú augmentáciu
- opisat celkovu triedu od Keras-u + Moja vlastne 2 triedy ktore som implementoval
- ako prebiehala augmentacia dat, opisat funkcie ktore sa pouzivaju pre dogenerovanie obrazkov a nazov tried ktore to implementuju.
- po vygenerovani obrazkov z Tomasovho nastroja, je potrebne spusti script convert.sh ktory prekonvertuje data do nizsej kvality
  a upravi pomenovanie obrazkov
- dat ukazku ako vyzera ten program a ako sa generuju data - stlacenim klavesi G

- Uvidim ci toto nedat radsej do navrhu
    * dat nejake priklady ako vyzeraju data
    * pozadie, 3D model zbrane, vysledok
    * ze bolo potrebne orezavat tieto obrazky, pomocou (link na web)

\subsection{Načitavanie dát}
- triedy ktore implementuju nacitavanie dat
- ako sa upravuju data pri nacitani

\subsection{Predspracovanie dát}
- preprocessing.py - Preprocessor, Preprocessing

\section{Modely}
- models.py - Models
- trieda ktore implementuje model
- ake informacie su potrebne pre vytvorenie modelu a ktore funkcie je potrebne definovat
- ktore triedy z Kerasu som pouzil a ktore triedy implementuju danu vrstvu, argumenty vrstiev ktore su nastavene a ako

\subsection{Ukladanie modelu}
- loader.py - DataSaver
- ktore informacie sa ukladaju
- ake funkcie implementuju ukladanie modelu
- opisat ukladanie modelu pre scikit-learn, ktora vec je na to pouzita, ze je prisposobene pre ukladanie velych matic
- opisat ukladanie modelu pre Keras, ktora vec to implementuje

\section{Trénovanie klasifikátorov}
- rozdelenie dat do 3 skupin - trenovanie data, validacne, testovacie
- cely postup trenovanie, asi to urobit rozdielne pre kazdy klasifikator osobitne
- obrazok trenovanie neuronovej siete
- Pridat ake optimalizatory a celkovo s akymi argumetmi spustat trenovanie, kolko epoch a pod...
- Zdovodnovat preco prave taketo nastavenie konvolucnej siete.
- pocet epoch bude potrebne zistit experimentalnych sposobom, alebo pouzit funkciu na save best of Keras-u

\subsection{Error angle funkcia}
- validacia uhla, vlastna funkcia (zdroj z kade to je)

\subsection{Hodnotenie presnosti klasifikatorov}
- opisat funkcie podľa ktorých sa ktorý klasifikátor bude hodnotiť
  mozno zobrat tabľku z druhej bakalarskej prace
- takisto, ktory script a ktore triedy implementuju evaluaciu modelu
- ako vyzera vystup po evaluaciu modelu

\section{Testovanie}
- testovanie siete priebiehalo pocas trenovanie ako validacia
  + na konci bola urobene evaluaciu modelu kde bol vysledok testovany

\section{Zhrnutie kapitoly}

% TODO


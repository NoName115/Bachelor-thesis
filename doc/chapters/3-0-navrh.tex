% ------------ NEW CHAPTER ------------
\chapter{Návrh riešenia}

V tejto kapitole si priblížime navrhované technológie pre riešienie problému, tejto práce.
Kapitola zahŕňa popis knižníc pre strojové učenie ako je Tensorflow a Keras, ktorý zovšeobecňuje a používa Tensorflow pre svoju prácu.
Následne nástroj scikit-learn, ktorý budeme používať pre jednoduchú klasifikáciu pomocou postupov, ktoré boli vysvetlené v kapitole \ref{chap:technologie}.





\begin{comment}

    \section{scikit-learn}
    \label{sec:scikitlearn}

    Scikit-learn je softvérová knižica pre strojové učenie pre programovací jazyk Python.
    Obsahuje množstvo algoritmov pre klasifikáciu, regresiu alebo zhlukovanie dát \cite{odkaz:scikitlearn}.
    Pre riešenie tejto práce obsahuje vhodné triedy, ktoré implementujú spomínané postupy zo sekcie \ref{sec:klasifikacia}.
    Príklady tried pre jednotlivé algoritmy:
    \begin{enumerate}
        \item[$\bullet$] \textbf{Nearest Neighbors} - scikit-learn poskytuje 2 rôzne klasifikátory pre algoritmus najbližsieho suseda.
        Trieda \textit{KNeighborsClassifier} klasifikuje na základe $k$ najbližšich susedov, kde $k$ je celé číslo špecifikované užívateľom.
        Druhá trieda \textit{RadiusNeighborsClassifier} implementuje klasifikáciu na základe počtu susedov v rámci pevného polomeru $r$ každého trénovacieho bodu,
        kde $r$ je hodnota s pohyblivou desatinou čiarkou určená užívateľom\footnote{\url{http://scikit-learn.org/stable/modules/neighbors.html\#nearest-neighbors-classification}}.
        \item[$\bullet$] \textbf{Support Vector Machines} - \textit{SVC}, \textit{NuSVC} a \textit{LinearSVC} sú triedy pre viac-triednu klasifikáciu.
        Pre ktoré je možné použit rôzne typy jadier[eng. kernels] \footnote{\url{http://scikit-learn.org/stable/modules/svm.html\#custom-kernels}}.
        \item[$\bullet$] \textbf{Stochastic Gradient Descent} - \textit{SGDClassifier} podporuje viac-triednu klasifikáciu pomocou kombinácie viacerých binárnych klasifikátorov v tzv.“one versus all” (OVA) schéme \footnote{\url{http://scikit-learn.org/stable/modules/sgd.html\#stochastic-gradient-descent}}.
    \end{enumerate}


    \section{Klasifikácia typu a určenie náklonu zbrane}
    Prvým bodom pri návrhu riešenia je výber programovacie jazyka a nástrojov, ktoré sú prispôsobné pre riešenie daného problému a tak uľahčujú výslednú implementáciu.
    Pre riešenie tejto práce bol vybraný programovací jazyk Python spolu s nástrojmi ktoré boli spomenúte už vyššie vid. sekcia \ref{sec:TensorflowKeras} a \ref{sec:scikitlearn}.
    Výsledny program bude mať za úlohu klasifikovať typ zbrane (krátka, dlhá) zo vstupného obrázku a následne určiť jej náklon.
    Pre klasifikáciu zbrane do daných kategórií sa ponúka niekoľko postupov klasifikácie, ktoré sú podrobne opísane v sekcíí \ref{sec:klasifikacia},
        pri určovaní náklonu zbrane v obraze môžeme použiť konvolučné neurónové siete.

\end{comment}

\subsection{navrh - poznamky}

\begin{enumerate}
    \item[$\bullet$] navrh pre 2 konvolucne neuronove siete, jedna inspirovane AlexNet druha VGG16,
    Zdovodnovat preco prave taketo nastavenie konvolucnej siete.
    Pridat ake optimalizatory a celkovo s akymi arugumetmi spustat trenovanie, kolko epoch a pod...
    \item[$\bullet$] spomenut ktore triedy zo scikit-learn chcem pouzit pre implementovanie klasifikatorov.
    Budem pouzivam SVM, Kmenas a Multilayer Perceptron
    \item[$\bullet$] urobit diagram celkoveho postupu ako to bude fungovat.
    Urobit 2 diagramy, 1.pre klasicky pristup, preprocessing --> Kmeans/SVM --> trenovanie --> vysledok 2.asi bude vyzerat rovnako.
    \item[$\bullet$] Opisat postup toho ako to budem robit, chcem vyskusat klasicky pristup HOG + SVM/KMeans
    \item[$\bullet$] Nasledne urobit len zakladny preprocessing pre NN a urobit 2 architektury CNN ktore porovnam
    \item[$\bullet$] napisat ktore trieda implementuje augmentaciu dat, ktore funkcie z akej veci chcem na co pouzit.
    Grayscale, HOG, zo scikit-image. CNN z Keras. Normalizacia NumPy. Augmentacia dat, 
    \item[$\bullet$] chcem vytvorit vlastnu funkciu pre kontrolu chyby pocas trenovanie - errorAngle
    \item[$\bullet$] Z kade navrhujem cerpat data pre zbrane, IMDB pre zbrane a ImageNet.
    Dalej od Tomasa nastroj chcem pouzit, ako sa tam generuju data, dat do navrhu screen-shot z toho.
    Opisat argumenty datagen v Keras ktore chcem pouzivat.
    \item[$\bullet$] Ktory programovaci jazyk chcem pouzit a k tomu aj frameworky, kde sa nasledne odvolat ktore triedy u nich co implementuju
    \item[$\bullet$] Z
\end{enumerate}

\subsection{implementacia - poznamky}
\begin{enumerate}
    \item[$\bullet$] Z kade som realne cerpal data nakoniec
    \item[$\bullet$] Ako prebiehala augmentacia dat, opisat funkcie ktore sa pouzivaju pre dogenerovanie obrazkov a nazov tried ktore to implementuju.
    \item[$\bullet$] 
\end{enumerate}

\subsection{vysledky - poznamky}

\begin{enumerate}
    \item[$\bullet$] vytvorit velku prehladnu tabulku pre budu kompletne vysledky, tak ako to je na git-e opisane
\end{enumerate}

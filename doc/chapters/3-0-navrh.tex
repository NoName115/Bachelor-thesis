% ------------ NEW CHAPTER ------------
\chapter{Návrh riešenia}
Kapitola bude zahŕnať použitý software pre tvorbu klasifikátorov a neurónových sieti, špecifikáciu hardwaru na ktorom bude trénovanie priebiehať.
Ďalej priblíži zbroje a spôsoby pre získanie, spracovanie a rozdelenie dát zbraní určených pre klasifikáciu typu a určenie náklonu v scéne.
Následne opíše celkový navrhovaný postup, od predspracovania obrazu až po zber výsledkov a určenie presnosti daných klasifikátorov.
Celá kapitola bude vychádzať z informácií ktoré boli spomenuté v kapitole \ref{chap:technologie}.



\section{Použitý software a hardware}
\label{sec:softwarehardware}
Prvým bodom pri návrhu riešenia je dôležité vybrať správny programovací jazyk a nástroje, ktoré sú pre riešenie daného problému vhodné
    a ktoré môžu uľahčiť aj celkovú implementáciu.

V kapitole \ref{sec:frameworks} bol vytvorený základný prehľad populárnych nástrojov pre tvorbu klasifikátorov, neurónových sietí a predspracovanie obrazu.
Vzhľadom na porovnanie týchto nástrojov, bude pre túto prácu použitý programovací jazyk Python, knižnica Keras pre tvorbu konvolučných neurónových sietí, ktorá
    ako svoj backend bude využívať knižnicu TensorFlow.
A následne pre klasický prístup ku klasifikácii obrázkov bude použitá knižnica Scikit-learn a Scikit-image pre implementáciu predspracovania obrazu.

Keďže trénovanie klasifikátorov a neurónových sietí je výpočtovo náročná operácia, je vhodné použiť výkonný hardware a akceleráciu výpočtov na GPU.
Pre túto prácu bude použitý počitač so štvorjadrovím procesorom i7-6700HQ, veľkosťou operačnej pamäte 16 GB a grafickou kartou Nvidia GeForce 940MX.
Na použitom počítači bude nainštalovaný 64-bitový operačný systém Fedora 27.



\section{Trénovacia databáza zbraní}
- Z kade navrhujem cerpat data pre zbrane, IMDB pre zbrane a ImageNet.
- Dalej od Tomasa nastroj chcem pouzit, ako sa tam generuju data, dat do navrhu screen-shot z toho.

\subsection{Generovanie dát z 3D modelov}
- Pouzitie tomasovho softwaru
- Program pre konvertovanie modelov do spravneho formatu

\subsection{Augmentácia obrázkov}
- Opisat argumenty datagen v Keras ktore chcem pouzivat.



\section{Klasifikácia typu zbrane}
Jedným z cieľou tejto práce je klasifikovať typy zbraní do 2 kategórií, a to na krátke a dlhé zbrane.
Táto klasifikácia môže prebiehať niekoľkými spôsobmi, prehľad týchto prístupov bol zhrnutý v kapitolách \ref{sec:detekcia} a \ref{sec:klasifikacia}.
Pre klasifikáciu v tejto práci bude použitých niekoľko z týchto prístupov a vo výsledku budú porovnané, ktorý dosiahol najlepšie výsledky.

Prvé riešenie bude spočívať v klasickom prístupe, ktoré pozostáva z predspracovania vstupných dát pomocou prekonvertovania dát do šedotónoveho obrazu a histogramu orientovaných prechodov (vid. \ref{sec:preprocessing}).
Na tieto predspracované dáta bude v jednom z riešení použitý K-Nearest-Neighbor klasfikátor, pre druhé riešenie bude použitý SVM klasifikátor, testovanie
    a porovnávanie výsledkov môže prebiehať s rôznymi konfiguráciami týchto klasifikátorov.
\begin{enumerate}
    \item[$\bullet$] Pre \textbf{K-Nearest-Neighbor}, knižnica scikit-learn poskytuje 2 rôzne implementácie tohto klasifikátora.
    Trieda \textit{KNeighborsClassifier} klasifikuje na základe $k$ najbližšich susedov, kde $k$ je celé číslo špecifikované užívateľom.
    
    Druhá implementácia je trieda \textit{RadiusNeighborsClassifier} ktorá klasifikuje na základe počtu susedov v rámci pevného polomeru $r$ každého trénovacieho bodu,
        kde $r$ je hodnota s pohyblivou desatinou čiarkou určená užívateľom\footnote{\url{http://scikit-learn.org/stable/modules/neighbors.html\#nearest-neighbors-classification}}.
    \item[$\bullet$] Pre \textbf{Support Vector Machines}, scikit-learn obsahuje 3 triedy \textit{SVC}, \textit{NuSVC} a \textit{LinearSVC} pre viac-triednu
        klasifikáciu s možnosťou použitia rôznych typov jadier \footnote{\url{http://scikit-learn.org/stable/modules/svm.html\#custom-kernels}}.
    %\item[$\bullet$] \textbf{Multi Layer Perceptron}
\end{enumerate}

Pre dalšie riešenie klasfikácie zbraní budú použité dve konvolučné neurónové siete, ktorých výsledky budú porovnané, podrobná architektúra sieti je opísana v kapitole \ref{sec:architekuraCNN}.
Predspracovanie vstupných dát bude pozostávať z normalizacie hôdnot RGB zložiek pixelov, nastavenia rovnomernej veľkosti strán obrázka (vid. \ref{sec:preprocessing})
    a následnej augmentácií dát pre zväčsenie počtu vstupných dát (vid. \ref{subsec:augmentacia}).



\section{Určenie náklonu zbrane}
- predspracovanie ake chcem pouzit, cely chain vysvetlit
- HOG, velkost okna, posun a pod...
- Opis vypoctu presnosti pre dany klasifikator
- Nasledne urobit len zakladny preprocessing pre NN a urobit 2 architektury CNN ktore porovnam

\subsection{Odchylka chyby}
- chcem vytvorit vlastnu funkciu pre kontrolu chyby pocas trenovanie - errorAngle



\section{Konvolučné neurónové siete}
\label{sec:architekuraCNN}

\subsection{Nastavenie parametrov}
\label{subsec:nastavenieparametrov}
Ako bolo spomínané v kapitole \ref{subsec:convolutionalneuralnetwork} je niekoľko parametrov, ktoré je potrebné
    nastaviť v konvolučných a pooling vrstvách.
Pre správne fungovanie siete je potrebné nastaviť aj správne hodnoty týchto parametrov, preto v navrhovaných architektúrach budú
    hodnoty parametrov nastavené na tie najpoužívanejšie.

Veľkosť výstupu a správnosť nastavenia parametrov v konvolučnej vrstve je možné výpočítať pomocou vzťahu:
\begin{equation}
    \frac{(W - F + 2*P)}{S} + 1
\end{equation}
Kde $W$ je veľkosť vstupných dát, $F$ je veľkosť filtra, $P$ je nastavenie zarovnania, v prípade nulového zarovnania je hodnota 1 a $S$ je veľkosť kroku.
Pre správné nastavenie musí byť výsledná hodnota celé číslo, kde táto výsledná hodnota udáva aj veľkosť výstupu.
Najpoužívanejšie hodnoty parametrov sú: $F = 3, S = 1, P = 1$ a vstup $W$ o hodnote, ktorá je deliteľná číslom 2 \cite{odkaz:CNNArchitecture}.

Funkciou pooling vrstvy je znižovanie dimenzionality a zároveň ponechanie dôležitých informácií, ako najpoužívanejšie hodnoty parametrov sú
    veľkosť filtra 2x2 s krokom 2 pri použití MaxPooling vrstvy \cite{odkaz:CNNArchitecture}.

% TODO batch-size, epochy
% CNN sa trenuje v cykloch nazyvanych epochy, v kazdej epoche prejdu cez sieť vsetky obrazky na ktorych sa sieť uci,
% pre ucenie je potrebne zvolit vhodny pocet tychro epoch aby sieť podala najelpsie vysledky a zaroven sa nepretrenovala.
% Preto je vhodne ukoncit trenovanie v bode kedy klesa presnosť siete na validacnych datach, v takom pripade to moze vies k pretrénovaniu siete
% na trenovacich datach.

\subsection{Návrh architektúr}
\label{subsec:navrharchitektur}
Prvý model siete je inšpirovaný architektúrou AlexNet (viď. \ref{subsec:popularCNN}).
Model obsahuje celkovo 15 vrstiev, z ktorých 4 je konvolučných, 4 max pooling, 5 dropout a 2 dense vrstvy.
Veľkosť vstupných dát do prvej vrstvy je 128x128x3.
Vo všetkých konvolučných vrstvách sú použité filtre o veľkosti 3x3 s krokom 1 a použitím nulového zarovnania.

V modeli sa každou konvolučnou vrstvou zdvojnásobuje počet filtrov, z počiatočných 16 až na 128 v poslednej vrstve.
Pooling vrstvy sú typu max, veľkosť filtra je 2x2 s posunom 2 po každej osi.
Za každou pooling vrstvou sa nachádza Droupout vrstva s nastavením 0.2, čiže 20 percent náhodných prepojení sa ignoruje.

Po štyroch blokoch konvolučnej, pooling a droupout vrstvy nasleduje dense vrstva s počtom prepojení 1024 a dropout vrstva s nastavním 0.5.
Ako posledná je dense vrstva s 2 alebo 72 prepojeniami a softmax klasifikátorom, počet výstupov závísí od toho, či určujeme typ alebo náklon zbrane.
V celej sietí sú použité ReLu aktivačné funkcie.

\begin{figure}[H]
    \centering
    \includegraphics[width=0.6\textwidth]{AlexNet_Like}
    \caption{AlexNet-Like navrhovaná architektúra.}
    \label{pic:kNN}
\end{figure}

% ----------- DRUHY NAVRH -----------

Druhý navrhovaný model je inšpirovaný architektúrou VGG sietí (viď. \ref{subsec:popularCNN}).
Vzhľadom na možnosti výkonu na ktorom bude trénovanie prebiehať, je navrhovaná sieť o dva bloky vrstviev menšia a taktiež konvolučne vrstvy obsahujú menej filtrov.

Celkovo sieť obsahuje 2 bloky obsahujúce 2 konvolučné vrstvy s počtom filtrov 32 a 64, pooling a droupout vrstvu s 20\% ignorovaním prepojení.
Ďalej nasledujú 3 konvolučné vrstvy so 128 filtrami a pooling vrstva.
Ako posledné sú 2 bloky obsahujúce dense vrstvu s počtom prepojení 2048 a dropout vrstvu s ignorovaním nastaveným na 50\%.
Posledná výstupná vrstva obsahuje 2 alebo 72 prepojení so softmax klasfikátorom.
Každá konvolučná vrstva obsahuje filtre o veľkosti 3x3, krokom 1 a s použitím nulového doplnku.
Pooling vrstvy sú typu max, veľkosť filtra je 2x2 s posunom 2 po každej osi.
V celej sietí je použitá ReLu aktivačná funkcia.

\begin{figure}[H]
    \centering
    \includegraphics[width=0.8\textwidth]{VGG_Like}
    \caption{VGG-Like navrhovaná architektúra.}
    \label{pic:kNN}
\end{figure}

\subsection{Hodnotenie presnosti modelov}
\label{subsec:hodnoteniepresnosti}
Hodnotenie presnosti modelov bude prebiehať pomocou dvoch metrík.

Prvá z metrík je tzv. chybová alebo tiež kontigenčná matica.
Každý stĺpec v matici predstavuje klasifikované triedy a jednotlivé riadky predstavujú správne triedy.
Tabuľka \ref{tab:chybovamatica} zobrazuje túto maticu.
Hodnota TP označuje počet správne klasifikovaných obrázkov triedy true, hodnota FP označuje počet nesprávne klasfikovaných obrázkov triedy true.
Hodnota TN označuje počet správne klasifikovaných obrázkov triedy false, hodnota FN označuje počet nesprávne klasifikovaných obrázkov triedy false \cite{odkaz:ChybovaMatica}.
\begin{table}[H]
    \centering
    \label{tab:chybovamatica}
        \begin{tabular}{lllc}
                                                                &                                   & \multicolumn{2}{c}{Klasifikované hodnoty}                                           \\ \cline{3-4} 
                                                                & \multicolumn{1}{l|}{}             & \multicolumn{1}{c|}{Trieda false}        & \multicolumn{1}{c|}{Trieda true}         \\ \cline{2-4} 
        \multicolumn{1}{c|}{\multirow{2}{*}{Správne hodnoty}} & \multicolumn{1}{c|}{Trieda false} & \multicolumn{1}{l|}{TN (True Negative)}  & \multicolumn{1}{c|}{FP (False Positive)} \\ \cline{2-4} 
        \multicolumn{1}{c|}{}                                 & \multicolumn{1}{c|}{Trieda true}  & \multicolumn{1}{l|}{FN (False Negative)} & \multicolumn{1}{c|}{TP (True Positive)}  \\ \cline{2-4} 
    \end{tabular}
    \caption{Chybová matica}
\end{table}
Pre prípad tejto práce si môžeme previesť triedu true na triedu krátke zbrane a triedu false na triedu dlhé zbrane.
Ako hodnotenie presnosti bude použitá metrika úspešnosť (angl. \textit{Accuracy}).

Úspešnosť - táto hodnota určuje ako často klasfikátor správne klasifikoval daný obrázok, počíta sa ako:
\begin{equation}
    Accuracy = \frac{TP + TN}{TP + TN + FP + FN}
\end{equation}

Ako druhá metrika, pre určenie presnosti modelov ktoré určujú náklon zbrane, bude implementovaná funkcia \textit{angle\_error}, ako bolo opísané v \ref{subsec:odchylkachyby}.
Hodnotenie funguje tak, že je nastavená prahová hodnota uhla podľa ktorej sa určuje, či daná predpoveď siete bola správna alebo nie.
Správne určenie je vypočítané, ako rozdiel medzi skutočným uhlom a predpovedaným uhol pomocou natrenovaného modelu, ak je rozdiel menší ako prahová hodnota, tak predikcia
    sa považuje za správnu v opačnom prípade za nesprávnu.


\section{Zhrnutie kapitoly}



\begin{comment}

    \section{Klasifikácia typu a určenie náklonu zbrane}
    Prvým bodom pri návrhu riešenia je výber programovacie jazyka a nástrojov, ktoré sú prispôsobné pre riešenie daného problému a tak uľahčujú výslednú implementáciu.
    Pre riešenie tejto práce bol vybraný programovací jazyk Python spolu s nástrojmi ktoré boli spomenúte už vyššie vid. sekcia \ref{sec:TensorflowKeras} a \ref{sec:scikitlearn}.
    Výsledny program bude mať za úlohu klasifikovať typ zbrane (krátka, dlhá) zo vstupného obrázku a následne určiť jej náklon.
    Pre klasifikáciu zbrane do daných kategórií sa ponúka niekoľko postupov klasifikácie, ktoré sú podrobne opísane v sekcíí \ref{sec:klasifikacia},
        pri určovaní náklonu zbrane v obraze môžeme použiť konvolučné neurónové siete.

\end{comment}

\begin{comment}

    \subsection{implementacia a vysledky- poznamky}
    \begin{enumerate}
        \item[$\bullet$] Z kade som realne cerpal data nakoniec
        \item[$\bullet$] Ako prebiehala augmentacia dat, opisat funkcie ktore sa pouzivaju pre dogenerovanie obrazkov a nazov tried ktore to implementuju.
        \item[$\bullet$] Obrazok trenovanie neuronovej siete
        \item[$\bullet$] Dosiahnute vysledky
        \item[$\bullet$] vytvorit velku prehladnu tabulku pre budu kompletne vysledky, tak ako to je na git-e opisane
    \end{enumerate}

\end{comment}


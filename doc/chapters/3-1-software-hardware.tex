
\section{Použitý software a hardware}
\label{sec:softwarehardware}
Prvým bodom pri návrhu riešenia je dôležité vybrať správny programovací jazyk a nástroje, ktoré sú pre riešenie daného problému vhodné
    a ktoré môžu uľahčiť aj celkovú implementáciu.

V kapitole \ref{sec:frameworks} bol vytvorený základny prehľad populárnych nástrojov pre tvorbu klasifikátorov, neurónových sieti a predspracovanie obrazu.
Vzhľadom na porovnanie týchto nástrojov, bude pre túto prácu použitý programovací jazyk Python, knižnica Keras pre tvorbu konvolučných neurónových sieti, ktorá
    ako svoj backend bude využívať knižnicu TensorFlow.
A následne pre klasický prístup ku klasifikácií obrázkov bude použitá knižnica Scikit-learn a Scikit-image pre implementáciu predspracovania obrazu.

Kedže trénovanie klasifikátorov a neurónových sieti je výpočetne náročna operácia, je vhodné použiť výkonný hardware a akceleráciu výpočtou na GPU.
Pre túto prácu bude použitý počitač so štvorjadrovím procesorom i7-6700HQ, veľkosťou operačnej pamäte 16 GB a grafickou kartou Nvidia GeForce 940MX.
Na použitom počitači bude nainštalovaný 64-bitový operačný systém Fedora 27.

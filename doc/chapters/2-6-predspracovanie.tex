
\section{Predspracovanie obrazu}
\label{sec:preprocessing}
Keďže riešenie bude obsahovať aj klasické prístupy pre klasifikáciu nie len riešenie pomocou konvolučných neurovnových sietí.
Tak je dôležité pred klasifikáciou vhodne upraviť resp. extrahovať z obrázka vhodné príznaky pre ich lepšiu klasifikáciu.

\subsubsection{Rovnomerný pomer strán}
Pre spracovanie obrázkov pomocou neurónových sietí je potrebné zabezpečiť, aby každý obrázok mal rovnakú veľkosť a rovnaký pomer strán.
Pretože väčšina modelov neurónovej siete predpokladá štvorcový vstupný obraz \cite{odkaz:NNPreprocessing}.

\subsubsection{Šedotónový obraz}
Ďalšou možnou technikou je vytvorenie šedotónového obrazu, kde všetky 3 zložky farby (RGB) zlúčime do jednej, šedotónovej.
Tento prevod sa vykoná výpočtom podľa vzorca:
\begin{equation}
    \label{eq:rgbtogray}
    Y = 0.2125*R + 0.7154*G + 0.0721*B \cite{prop:scikit-image}
\end{equation}
,kde $Y$ je výsledna hodnota sivého pixela, $R$ je hodnota červenej zložky pixela,
$G$ je zelená zložka pixela a $B$ je modrá zložka pixela v RGB \cite{odkaz:NNPreprocessing}.

\subsubsection{Rožšírenie vstupných dát}
Ďalšou bežnou technickou predspracovanie dát je rožšírenie existujúceho súboru dát s rôznymi variáciami existujúcich obrázkov.
Tieto variácie môžu zahŕňať zväčšovanie, zmenšovanie, rotačné zmeny a iné transformácie obrázkov.
Tento postup znižuje šancu, že neurónová sieť rozpozná nežiadúce charakteristiky v množine vstupných dát
	a taktiež sa zabezpečí zovšeobecnenie klasifikácie resp. rozpoznávania objektu \cite{odkaz:NNPreprocessing}.

\subsubsection{Normalizacia}
Hodnoty pixelov su väčšinou v rozpätí 0-255. Použivať tieto hodnoty na vstupe priemo do siete môže viesť k pretečeniu číselných premenných.
Taktiež sa ukázalo, že výber niektorých aktivačných funkcií nieje kompatibilný s takýmito vstupmi.
Nevhodný vstup môže viesť z nesprávnemu učeniu siete, preto je vhodné tieto dáta normalizovať do rozpätia 0-1 \cite{odkaz:PreprocessingNormalization}.

\subsubsection{Histogram orientovaných prechodov}
Základnou myšlienkou za dekriptorom histogramu orientovaných prechodov [eng. Histogram of oriented gradients descriptor] je to, že lokálny vzhľad objektu a tvar v obraze môžu byť popísané
	distribúciou intenzity gradientov alebo smerov hrán.

Obraz je rozdelený na malé spojené oblasti nazývane bunky [eng. cells], a pre pixely v každej bunke sa vytvára histogram orientovaných prechodov.
Deskriptor je spojenie týchto histogramov.
Pre zvýšenie presnosti môžu byť lokálne histogramy kontrastovo-normalizované výpočtom miery intenzity vo väčšej oblasti obrazu nazývanej blok a potom
	pomocou tejto hodnoty normalizovať všetky bunky v rámci bloku.
Táto normalizácia má za následok lepšiu invariáciu voči zmenám v osvetlení a tieňovaní v obraze \cite{prop:HOG}.

\begin{comment}
	\subsubsection{Priemerna štandartná odchýlka vstupných údajov}
	Je užitočné vytvoriť si tzv. ``stredný obraz'' získaný priemernými hodnotami pre každý pixel zo všetkých trénovacích dát.
	Timto spôsobom je možné vytvoriť si základny prehľad o štruktúre vstupných dát.
	Na základe toho môžeme potom do vstupným dát pridať rôzne dalšie variácie klasifikovaných objektov pre lepšie generalizovanie klasifikátora \cite{odkaz:NNPreprocessing}.
\end{comment}

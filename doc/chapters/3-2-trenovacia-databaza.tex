
\section{Trénovacia databáza zbraní}
Pre správne trénovanie klasifikátorov a neurónových sieti je dôležité mať dáta v dostatočnom počte a správne označené.
Trénovacia databáza je získana z niekoľkých zdrojov.
\begin{enumerate}
    \item[$\bullet$] \textbf{IMFDB}\footnote{\url{http://www.imfdb.org/wiki/Main_Page}} - je databáza záberov z filmov v ktorých sa nachádzajú zbrane.
    Obsahuje nie len celkové scény ale aj samostatné obrázky zbraní ktoré sa v danej scéne nachádzajú.
    Pre túto prácu je potrebné aby daný obrázok obsahoval iba zbraň, preto je potrebné obrázky z tejto databázy ručne odfiltrovať na samotné zbrane a scény so zbranami.
    Následne ich potom zaradiť do správnej kategórie na krátke a dlhé.
    \item[$\bullet$] \textbf{ImageNet}\footnote{\url{http://www.image-net.org/}} - je databáza obrázkov ktorá obsahuje viac ako 14 miliónov obrázkov vo viac ako 21000 kategóriach.
    Výhodou tejto databázy je že obrázku už patria do určenej kategórie a tak nieje potrebné ich ručné prechádzať a kategorizovať.
    \item[$\bullet$] \textbf{Google}\footnote{\url{http://www.google.com}} - pre doplnanie a zväčšenie počtu obrázkou je môžné použiť google vyhľadávanie.
    Tak ako v prípade IMFDB bude potrebné obrázky ručne kategorizovať.
    \item[$\bullet$] \textbf{Free3D}\footnote{\url{https://free3d.com/3d-models/weapons}} - databáza voľne dostupných 3D modelov zbraní s textúrami v rôzných formátoch.
\end{enumerate}

Databáza zbraní z prvých 3 zdrojov je možné použit na klasficikáciu zbraní do 2 kategórií na krátke a dlhé zbrane.
Pre trénovanie neurónových sieti na určenie náklonu zbrane v obraze bude potrebné použiť 3D modely zbraní a následne z nich vygenerovať
    obrázky zbraní v požadovaných natočeniach zbrane v obraze.

\subsection{Generovanie dát z 3D modelov}
Generovanie obrázkov pre určenie náklonu zbrane v scéne bude prebiehať pomocou databázy 3D modelov z posledného zdroja ktorý bol uvedení.
Následne sa použije software od Ing. Tomáša Goldmanna do ktorého je potrebné vložiť 3D model vo formáte g3db a pozadie scény.

Kedže 3D modely ktoré budú použité zo zdroja Free3D niesu v požadovanom formáte, je potrebné ich prekonvertovať, to je môžné urobiť
    pomocou nástroja \textbf{fbx-conv}\footnote{\url{https://github.com/libgdx/fbx-conv}}.
Tento nástroj podporuje konvertovanie 3D modelov z formátu fbx alebo obj to požadovaného formátu g3db.

\subsection{Augmentácia obrázkov}
Ako bolo opísane v kapitole \ref{sec:preprocessing} jedna z možností zväčšenia počtu dát na trénovanie je augmentácia vstupných dát.
V tejto práci môžeme rozdeliť augmentáciu dát do dvoch skupín, prvý druh augmentácie bude prebiehať pre trénovanie klasifikátorov ktoré
    určuju typ zbrane, druhý druh augmentácie bude prebiehať pri obrázkov určených na náklon zbrane.

Pri augmentácií dát pre klasifikovanie dvoch tried, na krátke a dlhé zbrane, prebehne niekoľko transfomácií obrázka.
\begin{enumerate}
    %\item[$\bullet$] \textbf{priblíženie} - priblíženie alebo oddialenie vstupného obrázku
    \item[$\bullet$] \textbf{Rotácia} - obrázky budú rotované o náhodny uhol v rozmedzí 0-180 stupňov.
    \item[$\bullet$] \textbf{Preklopenie} - vertikálne alebo horizonálne preklopenie obrázka.
    \item[$\bullet$] \textbf{Posun} - posun obrázka po x-ovej a y-ovej osi.
    \item[$\bullet$] \textbf{Výplň} - pri rotácií alebo posunu obrázku môže vzniknúť čierna plocha pixelov, tieto čierne pixely je potrbné vyplniť.
    Tieto čierne pixely budú doplnené hraničným pixelom obrázka ktorý bude nakopírovaný až po nový okraj.
\end{enumerate}

\begin{comment}
    - Opisat argumenty datagen v Keras ktore chcem pouzivat.
\end{comment}


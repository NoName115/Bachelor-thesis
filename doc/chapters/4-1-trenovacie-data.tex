
\section{Trénovacie dáta}
\label{sec:trenovaciedata}
- loader.py - DataLoader a DataSaver
- ake realne počty dát mam
- urobit tabulky ze ake typy obrazkov a pocty su pre ktoru vec
- ktore triedy implementujú augmentáciu
- opisat celkovu triedu od Keras-u + Moja vlastne 2 triedy ktore som implementoval
- ako prebiehala augmentacia dat, opisat funkcie ktore sa pouzivaju pre dogenerovanie obrazkov a nazov tried ktore to implementuju.
- po vygenerovani obrazkov z Tomasovho nastroja, je potrebne spusti script convert.sh ktory prekonvertuje data do nizsej kvality
  a upravi pomenovanie obrazkov
- dat ukazku ako vyzera ten program a ako sa generuju data - stlacenim klavesi G

- Uvidim ci toto nedat radsej do navrhu
    * dat nejake priklady ako vyzeraju data
    * pozadie, 3D model zbrane, vysledok
    * ze bolo potrebne orezavat tieto obrazky, pomocou (link na web)

- generovanie obrazkov z 3D modelov
    - Pre generovanie dat z 3D modelov bolo pouzitych 10 pozadi (dat obrazok pozadia) a 5 3D modelov, (obrazok ako cela scene vyzera),
      Takto bolo vygenerovanych X obrazkov pre otocenie o 2 stupne v kazdom smere.

\subsection{Načitavanie dát}
\label{subsec:nacitaniedat}
- triedy ktore implementuju nacitavanie dat
- ako sa upravuju data pri nacitani
- Pre nacitanie dat je implemntovane trieda... v scripte ... ktora nacita vstupne obrazky z viacerych moznosti, zo suboru,
  z priecinka kde pre klasifikaciu su urcene labely podla nazvu subfoldra a pre urcenie uhla sa urcuju z nazvu obrazka (diagram ako vyzera struktura foldrov)
  Nasledne je mozne este nacitat len jedne obrazok.

\subsection{Predspracovanie dát}
\label{subsec:predspracovaniedat}
- preprocessing.py - Preprocessor, Preprocessing


\section{Spracovanie obrazu}
\label{sec:imageprocessing}
Pre popis obrázkov a ostatných signálov sú často používané matematické modely.
Kde signál je funkcia závislá na určitých premenných s fyzikálnym významom, môže byť 1-dimenzionálna (napr. závisla na čase),
2-dimenzionálna (napr. obrázok závislý na 2 koordinátoch v ploche), 3-dimenzionálna (napr. popis pozície objektu v priestore), alebo aj viac-dimenzionálna \cite{book:ImageProcessing}.

Každý obraz môže byť definovaný ako spojitá funkcia s dvomi neznámymi $f(x,y)$ kde $x$ a $y$ sú súradnice v ploche.
Tento spojitý obraz je digitalizovaný na tzv. vzorkovacích miestach.
Tieto vzorkovacie miesta sú usporiadané v ploche, ich geometrický vzťah sa nazýva mriežka.
Digitálny obraz je potom dátova štruktúra, ktorá je bežne reprezentovaná ako matica.
Jeden bod v mriežke reprezentuje jeden element 2-dimenzionálneho obrazu nazývany pixel, v 3-dimenzionálnom obraze sá tento element nazýva voxel \cite{book:ImageProcessing}.
Pri viac-dimenzionálnych digitálnych obrazoch sa pri spracovaní obrazu používa vektor hodnôt(napr. RGB hodnoty obrazového bodu).

Oblasť spracovania digitálneho obrazu je v dnešnej dobe veľmi široká a nachádza uplatnenie vo viacerých oboroch.
Môže sa využívať pri automatickej vizuálnej inšpekcií produktou, pre zaistenie vyššej produktivity a kvality výrobku v továrňach.
Ďalej pri spracovaní snímkov z lietadiel alebo satelitov pre získanie dát o prírodnych zdrojoch, ako napr. v poľnohospodárstve alebo lesníctve.
Širokú aplikáciu má v medicíne pri obrázkoch získavaných pomocou röngenových zariadení, CT a magnetickej rezonancie \cite{book:ImageProcessingApplication}.
A v súčastnosti taktiež v automobilovom priemysle pri rozvýjajúcej sa oblati autonómneho riadenia automobilov.

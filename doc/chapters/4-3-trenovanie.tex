
\section{Trénovanie modelov}
\label{sec:trenovanie}
Pre trénovanie modelov sú v rámci implementácie použité všetky triedy ktoré boli podrobne opísane vyššie.
Proces trénovania prebieha v niekoľkých krokoch.
\begin{enumerate}
    \item[$\bullet$] Načitanie vstupných dát.
    \item[$\bullet$] Predspracovanie načitaných dát.
    \item[$\bullet$] Rozdelenie predspracovaných dát na trénovacie, validačné a testovanie v pomere 70\%:15\%:15\% pre konvolučné neurónove siete.
    A v pomere 70\%:30\% na trénovacie a testovanie pre K-means a SVM klasifikátor.
    \item[$\bullet$] Vytvorenie modelu.
    \item[$\bullet$] Spustenie trénovania vytvoreného modelu.
    \item[$\bullet$] Uloženie natrénovaného modelu a konfiguračného súboru.
    \item[$\bullet$] Testovanie presnosti modelu pomocou testovacích dát.
\end{enumerate}

Pre uľahčenie trénovania je implementovaný script \textit{train.py} ktorý požaduje niekoľko argumentov na vstupe.
Zoznam vstupných argumentov a ich význam, argumenty v hranatých zátvorkách su voliteľné:
\begin{enumerate}
  \item[$\bullet$] \textbf{--model} cesta k priečinku do ktorého bude natrénovaný model uložený.
  \item[$\bullet$] \textbf{--dataset} cesta k priečinku z ktorého budu načítane vstupné dáta.
  \item[$\bullet$] \textbf{--alg} voľba modelu ktorý bude trénovaný, tento argument má niekoľko možností.
  Možnosť ``cnnc'' pre trénovanie konvolučnej neurónovej siete určenej na klasifikáciu zbrane,
  ``cnna'' pre trénovanie konvolucnej neurónovej siete pre určenie náklonu zbrane a
  ``svm'' alebo ``kmeans'' pre trénovanie SVM alebo K-nearest-neighbor klasifikátora na určenie typu zbrane.
  \item[$\bullet$] \textbf{[--ep]} pri konvolučných neurónových sietach určuje počet epóch, pri jeho nezadaní je použitá
  základne hodnota 45.
  \item[$\bullet$] \textbf{[--bs]} určuje veľkosť batch-size pri trénovaní konvolučných neurónových sieti, jeho
  základna hodnota je 16.
  \item[$\bullet$] \textbf{[--rt]} je argument ktorý je potrebné zadať pri trénovani sieti na určenie náklonu zbrane,
  jeho hodnota určuje uhol náklonu na ktorý sa sieť bude učiť, možné hodnoty argumentu sú: ``yaw'', ``roll'', ``pitch''.
\end{enumerate}

\subsection{Hodnotenie presnosti modelov}
\label{subsec:hodnoteniepresnosti}
Pre celkové hodnotenie presnosti modelov bola implementovaná funkcia \textit{evaluate\_model} v scripte \textit{evaluation.py}.
Ktorá hodnotí presnosť sieťe podľa nižšie spomínaných metrík.

V prípade určenia typu zbrane sa používa tzv. chybová alebo tiež kontigenčná matica.
Kde každý stĺpec v matici predstavuje klasifikované triedy a jednotlivé riadky predstavujú správne triedy.
Tabuľka \ref{tab:chybovamatica} zobrazuje túto maticu.
Hodnota TP označuje počet správne klasifikovaných obrázkov triedy true, hodnota FP označuje počet nesprávne klasfikovaných obrázkov triedy true.
Hodnota TN označuje poćet správne klasifikovaných obrázkov triedy false, hodnota FN označuje počet nesprávne klasifikovaných obrazkov triedy false\cite{odkaz:ChybovaMatica}.
\begin{table}[H]
    \centering
    \label{tab:chybovamatica}
        \begin{tabular}{lllc}
                                                                &                                   & \multicolumn{2}{c}{Klasifikované hodnoty}                                           \\ \cline{3-4} 
                                                                & \multicolumn{1}{l|}{}             & \multicolumn{1}{c|}{Trieda false}        & \multicolumn{1}{c|}{Trieda true}         \\ \cline{2-4} 
        \multicolumn{1}{c|}{\multirow{2}{*}{Správne hodnoty}} & \multicolumn{1}{c|}{Trieda false} & \multicolumn{1}{l|}{TN (True Negative)}  & \multicolumn{1}{c|}{FP (False Positive)} \\ \cline{2-4} 
        \multicolumn{1}{c|}{}                                 & \multicolumn{1}{c|}{Trieda true}  & \multicolumn{1}{l|}{FN (False Negative)} & \multicolumn{1}{c|}{TP (True Positive)}  \\ \cline{2-4} 
    \end{tabular}
    \caption{Chybová matica}
\end{table}
Pre prípad tejto práce si môžeme preniesť triedu true na triedu krátke zbrane a triedu false na triedu dlhé zbrane.
Ako hodnotenie presnosti bude použitá metrika úspešnosť [eng. Accuracy].

Úspešnosť - táto hodnota určuje ako často klasfikátor správne klasifikoval daný obrázok, počíta sa ako:
\begin{equation}
    Accuracy = \frac{TP + TN}{TP + TN + FP + FN}
\end{equation}
Táto metrika je použitá aj počas trénovania sieti na predbežné určenie presnosti a taktiež pri koncovom testovaní modelu.

Avšak v prípade siete ktorá určuje náklon zbrane je potrebné túto metriku trošku pozmeniť.
Ako je opísane v \ref{subsec:odchylkachyby}, v scripte \textit{base.py} je implementovaná funkcia \textit{angle\_error} ktorá ráta priemerny rozdiel medzi
    skutočnymi a predpovedanými uhlami siete.
Táto funkcia je použita ako metrika počas trénovania siete, pri koncovom testovaní presnosti siete je nastavená prahová hodnota uhla podľa ktorej sa určuje,
    či daná predpoveď siete bola správna alebo nie.
Správne určenie je vypočítane ako rozdiel medzi skutočným uhlom a predpovedaným uhol pomocou siete, ak je rozdiel menší ako prehová hodnota, tak predikcia
    sa považuje za správnu.
Toto koncové testovanie siete už využíva metriku presnosti podľa chybovej matice.

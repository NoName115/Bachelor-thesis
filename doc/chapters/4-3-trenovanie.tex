
\section{Trénovanie klasifikátorov}
\label{sec:trenovanie}
Pre trénovanie modelov je implementovaný script train.py ktorý požaduje niekoľko argumentov pre spustenie.
Zoznam vstupných argumentov a ich význam, argumenty v hranatých zátvorkách su volitelné:
\begin{enumerate}
  \item[$\bullet$] \textbf{--model} cesta k priečinku do ktorého budú natrénované modely ukladané.
  \item[$\bullet$] \textbf{--dataset} cesta k priečinku z ktorého budu načítane všetky vstupné dáta.
  \item[$\bullet$] \textbf{--alg} voľba modelu ktorý bude trénovaný, tento argument na niekoľko možností.
  Možnosť ``cnnc'' pre trénovanie konvolučnej neurónovej siete určenej na klasifikáciu zbrane do 2 kategórií,
  ``cnna'' pre trénovanie konvolucnej neurónovej siete pre určenie náklonu zbrane,
  ``svm'' alebo ``kmeans'' pre trénovanie SVM alebo K-nearest-neighbor klasifikátora na určenie typu zbrane.
\end{enumerate}


- rozdelenie dat do 3 skupin - trenovanie data, validacne, testovacie
- cely postup trenovanie, asi to urobit rozdielne pre kazdy klasifikator osobitne
- obrazok trenovanie neuronovej siete
- Pridat ake optimalizatory a celkovo s akymi argumetmi spustat trenovanie, kolko epoch a pod...
- Zdovodnovat preco prave taketo nastavenie konvolucnej siete.
- pocet epoch bude potrebne zistit experimentalnych sposobom, alebo pouzit funkciu na save best of Keras-u
- pre zabezpecenie spravneho trenovanie boli pocty obrazkov rovnomerne pri trenovani

\subsection{Error angle funkcia}
\label{subsec:errorangle}
- validacia uhla, vlastna funkcia (zdroj z kade to je)

\subsection{Hodnotenie presnosti klasifikatorov}
\label{subsec:hodnoteniepresnosti}
- opisat funkcie podľa ktorých sa ktorý klasifikátor bude hodnotiť
  mozno zobrat tabľku z druhej bakalarskej prace
- takisto, ktory script a ktore triedy implementuju evaluaciu modelu
- ako vyzera vystup po evaluaciu modelu

% ------------ NEW CHAPTER ------------
\chapter{Návrh riešenia}

V tejto kapitole si priblížime navrhované technológie pre riešienie problému, tejto práce.
Kapitola zahŕňa popis knižníc pre strojové učenie ako je Tensorflow a Keras, ktorý zovšeobecňuje a používa Tensorflow pre svoju prácu.
Následne nástroj scikit-learn, ktorý budeme používať pre jednoduchú klasifikáciu pomocou postupov, ktoré boli vysvetlené v kapitole \ref{chap:technologie}.

\section{Tensorflow a Keras}
\label{sec:TensorflowKeras}

\subsubsection{Tensorflow}
TensorFlow je softvérová knižnica s otvoreným zdrojovým kódom[eng. open source library] pre numerické výpočty pomocou dátových vývojových diagramov[eng. data flow graphs].
Uzly v grafe reprezentujú matematické operácie, zatiaľ čo hrany grafu reprezentujú multidimenzionálne dátové polia (tensors), ktoré medzi sebou kominukujú.
Flexibilná architektúra umožňuje nasadenie na viacerých CPU alebo GPU, serveroch alebo aj mobilných zariadeniach.
TensorFlow bol pôvodne vyvinutý na účely výskumu strojového učenia a výskumu hlbokých neurónových sieti.
Systém je dostatočne všeobecný aby bol aplikovaný v mnohých iných oblastiach \cite{odkaz:TensorFlow}.

\subsubsection{Keras}
Vysoká všeobecnosť knižice TensorFlow pre jej širokú aplikáciu, robí jej použitie pre tvorbu neurónových sieti komplikovanejšiu.
Kde Keras, ktorý využíva TensorFlow ako svoj backend túto tvorbu neurónových sieti zjednodušil.
Za jeho backend je možné použiť aj CNTK alebo Theano.
Táto implementácia knižnice Keras robí experimentovanie s NN jednoduchšiu a rýchlejšiu.
Aj keď sa snaží implementáciu zjednodušiť, tak stále si zachováva modularitu a preto je možné dostatočne dobre modely NN upravovať a prisposobovať pre riešienie rôznych problémov \cite{odkaz:Keras}.

\section{scikit-learn}
\label{sec:scikitlearn}

Scikit-learn je softvérová knižica pre strojové učenie pre programovací jazyk Python.
Obsahuje množstvo algoritmov pre klasifikáciu, regresiu alebo zhlukovanie dát \cite{odkaz:scikitlearn}.
Pre riešenie tejto práce obsahuje vhodné triedy, ktoré implementujú spomínané postupy zo sekcie \ref{sec:klasifikacia}.
Príklady tried pre jednotlivé algoritmy:
\begin{enumerate}
    \item[$\bullet$] \textbf{Nearest Neighbors} - scikit-learn poskytuje 2 rôzne klasifikátory pre algoritmus najbližsieho suseda.
    Trieda \textit{KNeighborsClassifier} klasifikuje na základe $k$ najbližšich susedov, kde $k$ je celé číslo špecifikované užívateľom.
    Druhá trieda \textit{RadiusNeighborsClassifier} implementuje klasifikáciu na základe počtu susedov v rámci pevného polomeru $r$ každého trénovacieho bodu,
    kde $r$ je hodnota s pohyblivou desatinou čiarkou určená užívateľom\footnote{\url{http://scikit-learn.org/stable/modules/neighbors.html\#nearest-neighbors-classification}}.
    \item[$\bullet$] \textbf{Support Vector Machines} - \textit{SVC}, \textit{NuSVC} a \textit{LinearSVC} sú triedy pre viac-triednu klasifikáciu.
    Pre ktoré je možné použit rôzne typy jadier[eng. kernels] \footnote{\url{http://scikit-learn.org/stable/modules/svm.html\#custom-kernels}}.
    \item[$\bullet$] \textbf{Stochastic Gradient Descent} - \textit{SGDClassifier} podporuje viac-triednu klasifikáciu pomocou kombinácie viacerých binárnych klasifikátorov v tzv.“one versus all” (OVA) schéme \footnote{\url{http://scikit-learn.org/stable/modules/sgd.html\#stochastic-gradient-descent}}.
\end{enumerate}


\section{Klasifikácia typu a určenie náklonu zbrane}
Prvým bodom pri návrhu riešenia je výber programovacie jazyka a nástrojov, ktoré sú prispôsobné pre riešenie daného problému a tak uľahčujú výslednú implementáciu.
Pre riešenie tejto práce bol vybraný programovací jazyk Python spolu s nástrojmi ktoré boli spomenúte už vyššie vid. sekcia \ref{sec:TensorflowKeras} a \ref{sec:scikitlearn}.
Výsledny program bude mať za úlohu klasifikovať typ zbrane (krátka, dlhá) zo vstupného obrázku a následne určiť jej náklon.
Pre klasifikáciu zbrane do daných kategórií sa ponúka niekoľko postupov klasifikácie, ktoré sú podrobne opísane v sekcíí \ref{sec:klasifikacia},
    pri určovaní náklonu zbrane v obraze môžeme použiť konvolučné neurónové siete.

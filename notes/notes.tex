\documentclass[10pt,a4paper]{article}

\usepackage[left=1.5cm,text={18cm, 25cm},top=1cm]{geometry}
\usepackage[utf8]{inputenc}
\usepackage{geometry}
\usepackage{times}
\usepackage{float}
\usepackage{url}

\bibliographystyle{czplain}

\begin{document}

\title{Poznamky k bakalárke}
\author{Róbert Kolcún, xkolcu00@stud.fit.vutbr.cz}
\maketitle


\section{Úvod a motivácia}
Žvasty o tom koľko je útokov zbrani v ČR a pod...


\section{Technológie}

V tejto kapitole sa zoznámime so základnymi pojmami a princípmi, ktoré súvia s probletikou tejto práce a ktoré budú dalej využívané.
Kapitola vysvetľuje základne rozdelenia zbraní, princíp spracovania digitálneho obrazu a metódy jeho predspracovanie.
Ďalej sú tu vysvetlené rôzne spôsoby pre klasifikáciu týchto obrazových dát.

% http://scikit-learn.org/stable/tutorial/machine\_learning\_map/index.html
% 16008.pdf - Klasifikacne algoritmy - str.10+
% imagenet-classification-with-deep-convolutional-neural-networks.pdf - Introduction str.1+
% Rozpoznávání-termosnímků-obličeju.pdf
% 9614.pdf

\subsection{Zbrane 2.0}
Obvýklá definícia hovorí, že zbraň je nástroj, predmet, či dokonca celé zariadenie,
ktoré je prispôsobené k vyvolaniu ranivého účinku na živý organizmus alebo k ničeniu objektu\cite{book:StrelneZbrane}.
Za prvé zbrane môžeme považovať kopije ktoré používali luďia pri love zvierat asi pred 400,000 rokmi\cite{prop:SpearHistory}.

Vo všeobecnosti môžeme zbrane rozdeliť podľa množstvá krytérií, napr. podľa zdroja energie použitej k vypudeniu projektilu zo zbrane,
podľa konštrukcie a režimu streľby, ďalej z hľadiska postupu pri nabíjaní alebo podľa veku zbrane na nové - slúžiace svojmu účelu a historické - ktoré sú uź nespôsobilé k pôvodnemu účelu.
My sa zameriame na 2 základne rozdelenia a to poďla toho ako zbrane pôsobia na živú sílu[cz. živou sílu], delíme na:
\begin{enumerate}
	\item[$\bullet$] \textbf{Strelné} - rozrušujú vzdialený cieľ, živý alebo neživý, prodstredníctvom depadovej energie strely vypudenej zo zbrane.
	\item[$\bullet$] \textbf{Chladné} - účinkujú bodom alebo sekom naostrenej čepele, ktorá je vsadená do rukoväťe lebo je nasadená na tyč alebo porísko.
    \item[$\bullet$] \textbf{Úderné} - pôsobia na živý objekt tupým úderom s vojej časti, tkorá býva spojená s vhodnou rukoväťou.
\end{enumerate}
a podľa ovládateľnosti a možnosti prenášania ich delíme na:
\begin{enumerate}
	\item[$\bullet$] \textbf{Ručné} strelné zbrane môže prenášať a ovládať jediná osoba. Sú ovládané buď jednou rukou - krátke zbrane - alebo oboma rukami - dlhé zbrane.
	\item[$\bullet$] \textbf{Lafetované} zbrane musia byť vzhľadom ku svojej hmotnosti a rozmerom umiestnené na zvláštnom podstavci - \textit{lafete}. Takúto zbraň takisto väčsinou obsluhuje viac ľudí.
\end{enumerate}
V tejto práci sa zameramé hlavne na strelné, ručné zbrane.

\subsection{Spracovanie obrazu}
Pre popis obrázkov a ostatných signálov sú často používané matematické modely.
Kde signál je funkcia závislá na nejakých premenných s fyzikálnym významom, môže byť 1-dimenzionálna (napr. závisla na čase),
2-dimenzionálna (napr. obrázok závislý na 2 koordinátoch v ploche), 3-dimenzionálna (napr. popis pozície objektu v priestore), alebo aj viac-dimenzionálna\cite{book:ImageProcessing}.

    Každý obraz môže byť definovaný ako spojitá funkcia s dvomi neznámymi $$f(x,y)$$ kde $x$ a $y$ sú súradnice v ploche.
Tento spojitý obraz je digitalizovaný na tzv. vzorkovacích miestach.
Tieto vzorkovacie miesta sú usporiadané v ploche, ich geometrický vzťah sa nazýva mriežka.
Digitálny obraz je potom dátova štruktúra, ktorá je bežne reprezentovaná ako matica.
Jeden bod v mriežke reprezentuje jeden element 2-dimenzionálneho obraze nazývany pixel, v 3-dimenzionálnom obraze sá tento element nazýva voxel\cite{book:ImageProcessing}.
Pri viac-dimenzionálnych digitálnych obrazoch sa pri spracovaní obrazu používa vektor hodnôt(napr. RGB hodnoty obrazového bodu).

Oblasť spracovania digitálneho obrazu je v dnešnej dobe veľmi široká a nachádza uplatnenie vo viacerých oboroch.
Môže sa využívať pri automatickej vizuálnej inšpekcií produktou, pre zaistenie vyššej produktivity a kvality výrobku v továrňach.
Ďalej pri spracovaní snímkou z lietadiel alebo satelitou pre získanie dát o prírodnych zdrojoch, ako napr. v poľnohospodárstve alebo lesníctve.
Širokú aplikáciu má v medicíne pri obrázkoch získavaných pomocou röngenových zariadení, CT a magnetickej rezonancie\cite{book:ImageProcessingApplication}.
A v súčastnosti v automobilovom priemysle pri rozvýjajúcej sa oblati autonómneho riadenia automobilov.

\subsection{Klasifikácia}

\subsubsection{Nearest-Neighbor Classifiers}
- DP\_xbarne02.pdf

\subsubsection{Support Vector Machines}

\subsubsection{Stochastic Gradient Descent}

\subsubsection{Neural Network}
- 17150\_FULLTEXT.pdf - Background str.25, Deep learning str.30

- DiplomovaPraca.pdf - HLBOKÉ UČENIE A NEURÓNOVÉ SIETE str.20

- DP\_MajtánMartin.pdf - 1 NEURÓNOVÉ\ SIETE str.10

- FIT\_neuronky.pdf - Kap 3 Neuronové siete str.21

O čom je všeobecne Maching Learning
[Useful-things-about-machine-learning]

Machine Learning sa pouziva v širokej škále oblasti npr .....
My sme sa rozhodli použit ho na detekovanie typu zbrane a náklonu zbrane v obrazovej scéne.



\subsection{Decision Trees}
Asi NIE

http://scikit-learn.org/stable/modules/tree.html



\subsection{Predspracovanie obrazu}
http://scikit-image.org/docs/dev/auto\_examples/

\subsubsection{Hogova transformacia}
- F3-DP-2016-Erlebach-Jonas-Automaticka detekce pupily v obraze.pdf

\subsubsection{Detekcia hran}



\section{Návrh riešenia}

- Exploiting-the-complementary-strengths-of-multi-layer-CNN-image-retrieval.pdf,
preco pouzivat CNN na tento problem

\subsection{Keras}
- 17150\_FULLTEXT.pdf - Tensorflow and Keras str.26

- Navrhovana technologia pre ucenie

\subsection{scikit-learn}
- Kratky popis

\subsubsection{Databáza zbraní}
- ....

\subsection{Klasifikácia typu zbrane}

\subsection{Určenie natočenia zbrane}


\pagebreak
\section{Implementácia}

\subsection{Dataset}



% ------------- Bibliografy --------------
\pagebreak
\renewcommand\refname{Odkazy}
\bibliography{literatura}

\end{document}

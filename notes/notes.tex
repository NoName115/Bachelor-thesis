\documentclass[10pt,a4paper]{article}

\usepackage[left=1.5cm,text={18cm, 25cm},top=1cm]{geometry}
\usepackage[utf8]{inputenc}
\usepackage{geometry}
\usepackage{times}
\usepackage{float}
\usepackage{url}

\begin{document}

\title{Poznamky k bakalárke}
\author{Róbert Kolcún, xkolcu00@stud.fit.vutbr.cz}
\maketitle


\section{Úvod a motivácia}



\section{Extrakcia príznakov}



\section{Nearest Neighbor Classifier}

\subsection{Approximate Nearest Neighbor (ANN)}

\subsection{FLANN}
algorithms and libraries exist that can accelerate the nearest neighbor lookup in a dataset



\section{Rekurentné neurónové siete}
Branie predchadzajúceho výsledku v úvahu.



\section{Convolutional Neural Network classifer}

\subsection{VGG-16 classificator model}

\subsection{Classification model approach}

\subsubsection{Sliding window}

\subsubsection{Region proposal}

\subsection{Cascade classifiers}



\section{Stochastic Gradient Descent}
We use the Stochastic Gradient Descent (SGD),
which is commonly used in deep CNNs to update the weights



\section{Validacia dat}

\subsection{Rozdelenie trenovacich dat}

\subsection{Cross-validation}


\end{document}
